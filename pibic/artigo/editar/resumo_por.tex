% Substitua o texto abaixo pelo seu resumo.

A regra do impedimento é uma das mais polêmicas existentes num jogo de futebol, porém
não é comum a estudar a fundo sobre a mesma e discutir todos os cenários em que o
impedimento se configura, o mais conhecido é o cenário do atacante contra o penúltimo
defensor, é comum se pensar que precisa de apenas um defensor para que o adversário
não fique impedido, mas como em quase todas as situações de jogo o goleiro está
debaixo de suas traves, é fácil se esquecer que ele também conta como defensor e para a
regra, outro cenário é referente à linha da bola, quando o atacante já passou os 2 últimos
defensores, o mesmo pode realizar um passe para um companheiro que está na mesma
linha ou atrás da linha da bola e por último, o impedimento primeiro se configura na linha
de meio campo, se um jogador ruma ao ataque e se encontra antes da linha do meio
campo, qualquer passe em sua direção pode ser realizado, entretanto, se os 2 últimos
defensores estiverem no campo de ataque e o atacante estiver após a linha do meio de
campo, qualquer bola direcionada ao mesmo será considerado impedimento. A leitura e
base utilizada para este trabalho serão de pesquisas consolidadas e que já conseguiram
reproduzir a proposta com sucesso ou que serviram de base para outros trabalhos na
área e tem validade para este.



% Substitua as palavras-chave abaixo pelas suas.


 \textbf{Palavras-chave}: Arbitragem esportiva; Impedimento; Inteligência Artificial; Processamento Digital; Visão Computacional.

