% ----------------------------------------------------------
% Introdução (exemplo de capítulo sem numeração, mas presente no Sumário)
% ----------------------------------------------------------
%\chapter*[Introdução]{Introdução}
\chapter{Introdução}
%\addcontentsline{toc}{chapter}{INTRODUÇÃO}
% ----------------------------------------------------------
A aplicação de técnicas de verificação formal em sistemas baseados em Inteligência Artificial (IA) tem se tornado crucial para garantir a confiabilidade desses modelos em ambientes críticos. Este trabalho apresenta uma abordagem utilizando o ESBMC (\textit{Efficient SMT-Based Context-Bounded Model Checker}) para a verificação de propriedades de segurança em redes neurais e sistemas de controle neuro-simbólicos.

\chapter{Objetivos}

\section{Geral}

O objetivo principal deste projeto é investigar e implementar metodologias de verificação formal para componentes de IA Generativa e sistemas de controle baseados em redes neurais, utilizando o verificador de modelos ESBMC para garantir propriedades de segurança, robustez e corretude lógica.

\section{Específicos}

Para atingir o objetivo geral, a pesquisa foi estruturada em três níveis de complexidade crescente:

\begin{itemize}
    \item \textbf{Nível 1 — Verificação de Segurança em Redes Neurais}: Desenvolver modelos em C e Python para componentes básicos de redes neurais, garantindo que a saída do modelo permaneça dentro de uma região segura (\textit{post-condition}) para um conjunto de entradas delimitado. Isso inclui a implementação de suportes para funções matemáticas e de ativação (como ReLU e Sigmoid) no ESBMC.
    
    \item \textbf{Nível 2 — Verificação de Controladores Neurais}: Expandir a análise para redes neurais especificamente projetadas para controle de sistemas dinâmicos, verificando propriedades locais de corretude e segurança do código do controlador.
    
    \item \textbf{Nível 3 — Verificação de Propriedades de Sistema (Malha Fechada)}: Realizar a verificação formal de propriedades de alto nível do sistema completo (planta + controlador neural). Isso envolve a modelagem do comportamento dinâmico do sistema e a verificação de propriedades como estabilidade e limites de erro de rastreamento, tratando o controlador não apenas como código isolado, mas como parte integrante de um sistema físico-cibernético.
\end{itemize}
