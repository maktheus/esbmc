\chapter{Fundamentação Teórica}
\label{cap:fundamentacao}

Este capítulo apresenta os conceitos fundamentais que sustentam a verificação formal de software, com foco em Verificação de Modelos Limitada (\textit{Bounded Model Checking} - BMC) e na arquitetura do verificador ESBMC.

\section{Verificação Formal e Model Checking}

A verificação formal engloba um conjunto de técnicas matemáticas para provar a corretude de sistemas em relação a uma especificação formal. Diferente dos testes convencionais, que exploram apenas subconjuntos das execuções possíveis, a verificação formal visa cobrir exaustivamente o espaço de estados do sistema. 

O \textit{Model Checking} é uma técnica automática que verifica se um modelo de um sistema satisfaz uma determinada propriedade lógica (geralmente expressa em lógica temporal). Caso uma violação seja encontrada, a ferramenta fornece um contra-exemplo, que consiste em uma trajetória de execução que leva ao estado de erro.

\section{Bounded Model Checking (BMC)}

O \textit{Bounded Model Checking} é uma técnica baseada em satisfatibilidade lógica (SAT/SMT) onde o sistema é analisado até um limite fixo de passos (\textit{bound} $k$). Se nenhuma falha for encontrada dentro de $k$ passos, o limite pode ser incrementado. O processo envolve:
\begin{enumerate}
    \item Desenrolamento dos laços e chamadas de função até o limite $k$;
    \item Conversão do código e das propriedades em uma fórmula lógica;
    \item Verificação da fórmula por um \textit{solver} SMT.
\end{enumerate}

\section{O Verificador ESBMC}

O ESBMC (\textit{Efficient SMT-Based Context-Bounded Model Checker}) é um verificador de código aberto que estende o BMC para programas multicore e sistemas dinâmicos. Ele suporta linguagens como C, C++, Python e CUDA. 

O fluxo de trabalho do ESBMC consiste em:
\begin{itemize}
    \item \textbf{Frontend}: Análise léxica e sintática do código-fonte;
    \item \textbf{GOTO-Program}: Transformação em um grafo de fluxo de controle simplificado;
    \item \textbf{Execução Simbólica}: Geração de equações que representam a semântica do programa;
    \item \textbf{SMT Encoding}: Codificação das equações em fórmulas para \textit{solvers} como Z3, Bitwuzla ou Boolector.
\end{itemize}

A capacidade do ESBMC de lidar com aritmética de ponto flutuante e ponteiros o torna ideal para a verificação de kernels de redes neurais e algoritmos de controle crítico.
